\documentclass{article}
\usepackage[spanish]{babel}
\usepackage[utf8]{inputenc}
\usepackage[left=3.5cm,right=3.5cm,top=2cm,bottom=2cm]{geometry}
\usepackage{amsmath,amssymb}
\decimalpoint

\title{{\LARGE Procesos Estocásticos}\\
{\Large \textbf{Tarea 2. Elementos de Cadenas de Markov}}}
\author{}
\date{\textit{30 de Marzo de 2022}}
\begin{document}
\maketitle
\begin{enumerate}
    \item Prueba que la probabilidad de que \emph{exactamente} uno de los dos eventos $A$ y $B$ ocurra es :
    $$
    P(A)+P(B)-2P(A\cap B)
    $$
    Prueba que:
    $$
    P(A\cup B\cup C) = 1-P(A^c\vert B^c\cap C^c)P(B^c\cap C^c)P(C^c)
    $$
    \item Supongamos que $(\Omega,\mathcal{F},P)$ es un espacio de probabilidad y $B\mathcal{F}$ es tal que $P(B)>0$. Sea $Q:\mathcal{F}\rightarrow [0,1]$ definida como $Q(A) = P(A\vert B)$. Muestra que $(\Omega, \mathcal{F},Q)$ es un espacio de probabilidad. Si $C\in\mathcal{F}$ y $Q(C)>0$, muestra que $Q(A\vert C) = P(A\vert B\cap C)$. ¿Cómo interpretas esta último resultado?
    \item Una cadena de Markov, tiene la matriz de transición:
    $$
    \mathbb{P} = \bordermatrix{
          & 0   & 1   & 2   \cr
        0 & 0.7 & 0.2 & 0.1 \cr
        1 & 0   & 0.6 & 0.4 \cr
        2 & 0.5 & 0   & 0.5 \cr
    }
    $$
    Determina las probabilidades:
    $$
    P(X_3 = 1, X_2 = 1 \vert X_1=0) \quad \mbox{ y }\quad P(X_2 = 1, X_1 = 1\vert X_0 = 0)
    $$
    \item Demuestra que si $\{X_n\}$ es una cadena de Markov, entonces:
    $$
    P(X_0=i_0\vert X_1 =i_1,X_2=i_2 \ldots,X_n=i_n) = P(X_0=i_0\vert X_1 =i_1)
    $$
    \item Supongamos que tenemos dos cajas con $2d$ bolas entre ambas. De estas bolas, $d$ son negras y $d$ son rojas. Inicialmente colocamos $d$ bolas en la caja uno y d bolas en la caja dos. En cada ensayo, elegimos aleatoriamente una bola de la caja uno y una bola de la caja dos y las intercambiamos de caja. Denotamos con $X_n$ al número de bolas negras en la caja uno después de $n$ ensayos. Encuentra las probabilidades de transición para la cadena de Markov $\{X_n\}$ y escribe la matriz de transición.
    \item Una cadena de Markov tiene matriz de transición:
    $$
    \mathbb{P} = \bordermatrix{
          & 1     & 2    & 3   & 4   \cr
        1 & 1/2   & 1/2  & 0   & 0   \cr
        2 & 0     & 1/3  & 1/3 & 1/3 \cr 
        3 & 1/4   & 1/4  & 1/4 & 1/4 \cr 
        4 & 0     & 0	 & 1/2 & 1/2 \cr 
    }
    $$
    y distribución inicial:
    $$
    \pi = (1/2, 1/4,1/8, 1/8)
    $$
    Calcula $\pi_n(i) = P(X_n = i)$, para $i\in \{1,2,3,4\}$ y  $n=1,2,3,4$. Calcula $P(X_4\neq X_1)$. Calcula $P(X_5 = 3\vert X_3 = 1,max(X_0,X_1,X_2)<3)$
\end{enumerate}
\end{document}