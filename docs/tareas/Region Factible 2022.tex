\documentclass[12pt]{article}
\usepackage[spanish]{babel}
\usepackage[utf8]{inputenc}
\usepackage{pgfplots}
\pgfplotsset{compat=1.15}
\usepackage{mathrsfs}
\usetikzlibrary{arrows}
\usepackage{amsmath,amssymb}
\setlength{\oddsidemargin}{0.5cm}
\setlength{\evensidemargin}{0.5cm}
\setlength{\textwidth}{15cm}
\setlength{\topmargin}{-2cm}
\setlength{\textheight}{22cm}
%\usepackage{tgbonum}
%Numbered environment
\newcounter{example}[section]
\newenvironment{example}[1][]{\refstepcounter{example}\par\medskip
   \noindent \textsf{Ejemplo~\theexample. #1}\par\medskip\noindent\hrule\par\medskip\rmfamily}{\par\medskip\hrule\par\medskip}
\usetikzlibrary{babel}

\begin{document}

\begin{example}[Optimo único]
    Consideremos un problema de programación lineal con función objetivo y restricciones:
    \begin{alignat*}{5}
     max~&  &  2&x_1- &  &x_2 &  &       &      &      \\
    s.a.~&  &   &x_1+ &  &x_2 &  &\leq 4 &      &      \\
         &  &  -&x_1+ &  &x_2 &  &\leq 2 &      &      \\
         &  &   &x_1- & 2&x_2 &  &\leq 2 &      &      \\
         &  &   &     &  &x_i &  &\geq 0 & \quad&i=1,2.
    \end{alignat*}

%\definecolor{ffxfqq}{rgb}{1,0.4980392156862745,0}
\begin{tikzpicture}[line cap=round,line join=round,>=triangle 45,x=1cm,y=1cm]
    \begin{axis}[
            x=1cm,y=1cm,
            axis lines=middle,
            %ymajorgrids=true,
            %xmajorgrids=true,
            xmin=-7.5,
            xmax=5,
            ymin=-6.5,
            ymax=5.0238875720826925,
            xtick={-10,-9,...,16},
            ytick={-6,-5,...,5},]
        \clip(-7.5,-6.5) rectangle (5,5);
        \fill[line width=0pt,color=gray,fill=gray,fill opacity=0.22] (1,3) -- (0,2) -- (0,0) -- (2,0) -- (3.3333333333333335,0.6666666666666666) -- cycle;
        %
        \draw [line width=2pt,domain=-10.576880237783161:16.926706233411924] plot(\x,{4-\x});
        \draw [line width=2pt,domain=-10.576880237783161:16.926706233411924] plot(\x,{2+\x});
        \draw [line width=2pt,domain=-10.576880237783161:16.926706233411924] plot(\x,{(-2+1*\x)/2});
        %\draw [line width=2pt,domain=-10.576880237783161:16.926706233411924] plot(\x,{(-2--1*\x)/2});
        \draw [line width=2pt,color=blue,domain=-10.576880237783161:16.926706233411924] plot(\x,{-2+2*\x});
        \draw [-stealth,line width=2pt,color=blue] (1.5,1) -- (2.25,0.625);
        \draw [line width=1.6pt,dotted,color=blue,domain=-10.576880237783161:16.926706233411924] plot(\x,{(-12.5--5*\x)/2.5});
        \begin{scriptsize}
            \draw [fill=red] (3.3333333333333335,0.6666666666666666) circle (3pt);
            \draw[color=red] (3.4808969541054897,0.9973856294943558) node {$x^*$};
            \draw[color=blue] (2.4,0.9) node {$\nabla z_0$};
        \end{scriptsize}
    \end{axis}
\end{tikzpicture}

Este problema tiene una única solución óptima en el punto:
$$
x^* = \left(
    \begin{matrix}
    10/3\\[3pt]
    2/3
    \end{matrix}
    \right)
$$
Con valor  óptimo $z_0 =cx^* = 6$.
\par
\end{example}%
\newpage
\begin{example}[Segmento óptimo]
    Consideremos un problema de programación lineal con función objetivo y restricciones:
    \begin{alignat*}{5}
     min~&  &   &x_1- &  &x_2 &  &       &      &      \\
    s.a.~&  &   &x_1+ &  &x_2 &  &\leq 4 &      &      \\
         &  &  -&x_1+ &  &x_2 &  &\leq 2 &      &      \\
         &  &   &x_1- & 2&x_2 &  &\leq 2 &      &      \\
         &  &   &     &  &x_i &  &\geq 0 & \quad&i=1,2.
    \end{alignat*}

\begin{tikzpicture}[line cap=round,line join=round,>=triangle 45,x=1cm,y=1cm]
    \begin{axis}[
            x=1cm,y=1cm,
            axis lines=middle,
            %ymajorgrids=true,
            %xmajorgrids=true,
            xmin=-7.5,
            xmax=5,
            ymin=-6.5,
            ymax=5.0238875720826925,
            xtick={-10,-9,...,16},
            ytick={-6,-5,...,5},]
        \clip(-7.5,-6.5) rectangle (5,5);
        \fill[line width=0pt,color=gray,fill=gray,fill opacity=0.22] (1,3) -- (0,2) -- (0,0) -- (2,0) -- (3.3333333333333335,0.6666666666666666) -- cycle;
        %
        \draw [line width=2pt,domain=-10.576880237783161:16.926706233411924] plot(\x,{4-\x});
        \draw [line width=2pt,domain=-10.576880237783161:16.926706233411924] plot(\x,{2+\x});
        \draw [line width=2pt, color=red,domain=0:1] plot(\x,{2+\x});
        \draw [line width=2pt,domain=-10.576880237783161:16.926706233411924] plot(\x,{(-2+1*\x)/2});
        %\draw [line width=2pt,domain=-10.576880237783161:16.926706233411924] plot(\x,{(-2--1*\x)/2});
        \draw [line width=2pt,color=blue,domain=-10.576880237783161:16.926706233411924] plot(\x,{-1+\x});
        \draw [-stealth,line width=2pt,color=blue] (1.5,0.5) -- (1.9,0.1);
        \draw [line width=1.6pt,dotted,color=blue,domain=-10.576880237783161:16.926706233411924] plot(\x,{0.5+\x});
        \begin{scriptsize}
            \draw [fill=red] (0,2) circle (3pt);
            \draw [fill=red] (1,3) circle (3pt);
            \draw[color=red] (0.2,1.8) node {$x^*_0$};
            \draw[color=red] (1.4,3) node {$x^*_1$};
            \draw[color=blue] (2.2,0.3) node {$\nabla z_0$};
        \end{scriptsize}
    \end{axis}
\end{tikzpicture}

La solución óptima es el segmento: $\lambda x^*_0 + (1-\lambda) x_1^*$, con $\lambda \in [0,1]$ y 
$$
x^*_0 = \left(
    \begin{matrix}
    0\\[3pt]
    2
    \end{matrix}
    \right),
\qquad
x^*_1 = \left(
    \begin{matrix}
    1\\[3pt]
    3
    \end{matrix}
    \right)
$$
Con valor óptimo $z_0 = c(\lambda x_0^* + (1-\lambda)x_1^*)=-2$, en particular: $z_0 = cx_0^* = cx_1^*=-2$

\end{example}%
\newpage


\begin{example}[Rayo óptimo]
    Consideremos un problema de programación lineal con función objetivo y restricciones:
    \begin{alignat*}{5}
     min~&  &   &x_1- &  &x_2 &  &       &      &      \\
    s.a.~&  &   &x_1+ &  &x_2 &  &\geq 4 &      &      \\
         &  &  -&x_1+ &  &x_2 &  &\leq 2 &      &      \\
         &  &   &x_1- & 2&x_2 &  &\leq 2 &      &      \\
         &  &   &     &  &x_i &  &\geq 0 & \quad&i=1,2.
    \end{alignat*}

\begin{tikzpicture}[line cap=round,line join=round,>=triangle 45,x=1cm,y=1cm]
    \begin{axis}[
            x=1cm,y=1cm,
            axis lines=middle,
            %ymajorgrids=true,
            %xmajorgrids=true,
            xmin=-7.5,
            xmax=5,
            ymin=-6.5,
            ymax=5.0238875720826925,
            xtick={-10,-9,...,16},
            ytick={-6,-5,...,5},]
        \clip(-7.5,-6.5) rectangle (5,5);
        \fill[line width=0pt,color=gray,fill=gray,fill opacity=0.22] (3,5) -- (1,3) -- (3.3333333333333335,0.6666666666666666) -- (5,1.5) -- (5,5) -- cycle;
        %
        \draw [line width=2pt,domain=-10.576880237783161:16.926706233411924] plot(\x,{4-\x});
        \draw [line width=2pt,domain=-10.576880237783161:16.926706233411924] plot(\x,{2+\x});
        \draw [line width=2pt, color=red,domain=1:3] plot(\x,{2+\x});
        \draw [line width=2pt,domain=-10.576880237783161:16.926706233411924] plot(\x,{(-2+1*\x)/2});
        %\draw [line width=2pt,domain=-10.576880237783161:16.926706233411924] plot(\x,{(-2--1*\x)/2});
        \draw [line width=2pt,color=blue,domain=-10.576880237783161:16.926706233411924] plot(\x,{-1+\x});
        \draw [-stealth,line width=2pt,color=blue] (1.5,0.5) -- (1.9,0.1);
        \draw [line width=1.6pt,dotted,color=blue,domain=-10.576880237783161:16.926706233411924] plot(\x,{0.5+\x});
        \draw [-stealth,line width=2pt,color=red] (1,3) -- (1.5,3.5);
        \begin{scriptsize}
            %\draw [fill=red] (0,2) circle (3pt);
            \draw [fill=red] (1,3) circle (3pt);
            %\draw[color=red] (0.2,1.8) node {$z_0$};
            \draw[color=red] (1.4,3) node {$x^*$};
            \draw[color=blue] (2.2,0.3) node {$\nabla z_0$};
        \end{scriptsize}
    \end{axis}
\end{tikzpicture}

La solución óptima de este problema es el rayo óptimo: $x^*+\mu d$ con $\mu\geq 0$ y 
$$
x^* = \left(
    \begin{matrix}
    1\\[3pt]
    3
    \end{matrix}
    \right),
\qquad
d = \left(
    \begin{matrix}
    1/2\\[3pt]
    1/2
    \end{matrix}
    \right)
$$
Con valor óptimo $z_0=c(x^*+\mu d)=-2$, observemos que $z_0= cx^*$ y $cd =0$.
\end{example}
\newpage

\begin{example}[Optimo no finito]
    Consideremos un problema de programación lineal con función objetivo y restricciones:
    \begin{alignat*}{5}
     max~&  &  2&x_1- &  &x_2 &  &       &      &      \\
    s.a.~&  &   &x_1+ &  &x_2 &  &\geq 4 &      &      \\
         &  &  -&x_1+ &  &x_2 &  &\leq 2 &      &      \\
         &  &   &x_1- & 2&x_2 &  &\leq 2 &      &      \\
         &  &   &     &  &x_i &  &\geq 0 & \quad&i=1,2.
    \end{alignat*}

\begin{tikzpicture}[line cap=round,line join=round,>=triangle 45,x=1cm,y=1cm]
    \begin{axis}[
            x=1cm,y=1cm,
            axis lines=middle,
            %ymajorgrids=true,
            %xmajorgrids=true,
            xmin=-7.5,
            xmax=5,
            ymin=-6.5,
            ymax=5.0238875720826925,
            xtick={-10,-9,...,16},
            ytick={-6,-5,...,5},]
        \clip(-7.5,-6.5) rectangle (5,5);
        \fill[line width=0pt,color=gray,fill=gray,fill opacity=0.22] (3,5) -- (1,3) -- (3.3333333333333335,0.6666666666666666) -- (5,1.5) -- (5,5) -- cycle;
        %
        \draw [line width=2pt,domain=-10.576880237783161:16.926706233411924] plot(\x,{4-\x});
        \draw [line width=2pt,domain=-10.576880237783161:16.926706233411924] plot(\x,{2+\x});
        \draw [line width=2pt,domain=-10.576880237783161:16.926706233411924] plot(\x,{(-2+1*\x)/2});
        %\draw [line width=2pt,domain=-10.576880237783161:16.926706233411924] plot(\x,{(-2--1*\x)/2});
        \draw [line width=2pt,color=blue,domain=-10.576880237783161:16.926706233411924] plot(\x,{-2+2*\x});
        \draw [-stealth,line width=2pt,color=blue] (1.5,1) -- (2.25,0.625);
        \draw [line width=1.6pt,dotted,color=blue,domain=-10.576880237783161:16.926706233411924] plot(\x,{(-12.5--5*\x)/2.5});
        \begin{scriptsize}
            \draw[color=blue] (2.4,0.9) node {$\nabla z_0$};
        \end{scriptsize}
    \end{axis}
\end{tikzpicture}

Este problema no tiene solución óptima finita.
\end{example}


\end{document}