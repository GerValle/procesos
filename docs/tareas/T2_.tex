\documentclass{report}
    \usepackage[utf8]{inputenc}
    \usepackage[spanish]{babel}
    \usepackage{amsmath}
    \usepackage{graphics}
    \usepackage{amssymb}
    \setlength{\oddsidemargin}{0.5cm}
    \setlength{\evensidemargin}{0.5cm}
    \setlength{\textwidth}{15cm}
    \setlength{\topmargin}{-2cm}
\begin{document}
\begin{center}
    \textsf{\Large Procesos Estocásticos}
    \par\medskip
    \textsf{\large Tarea 2}
    \end{center}
    \hrule
    \par\bigskip

De los problemas a continuación, deberás entregar los problemas 2, 3, 4, 6 y 8.
\begin{enumerate}
    \item Sea $\left\{X_n\right\}$ una cadena de Markov con espacio de estados $S = \left\{1,2,3\right\}$ y matriz de transición:
    $$P= \left[
        \begin{matrix}
            0.6 & & 0.3 & 0.1 \\
            0.3 & & 0.3 & 0.4 \\
            0.4 & & 0.1 & 0.5 \\
        \end{matrix}
        \right]
    $$

Si se sabe que el proceso empieza en $X_0=1$, determina la probabilidad $P\left(X_0=1,X_1=0, X_2 = 2 \right)$.
\item Considera el problema de enviar un mensaje binario 0,1, a través de un canal de señales que consiste de varias estaciones, donde la transmisión a través de cada estación esta sujeta a una probabilidad fija $\alpha$ de error. Supongamos que $X_0=0$ y que $X_n$ es la señal recibida en la estación $n$.  Supongamos que $X_n$ es una cadena de Markov con probabilidades de transición $P(0,0) = P(1,1)= 1-\alpha$ y $P(0,1)=P(1,0)= \alpha$, con $0<\alpha<1$.
\begin{enumerate}
    \item Determina la probabilidad $P\left(X_0=0,X_1=0, X_2=0\right)$
    \item Determina la probabilidad de que la señal recibida en la estacion 2 sea correcta.
\end{enumerate}
\item Las variables aleatorias $\xi_1, \xi_2, \ldots$ son independientes distribuidas:
$$
P\left(X_n=k\right)=
\begin{cases}
    0.1 \text{ si } k = 0 \\
    0.3 \text{ si } k = 1 \\
    0.2 \text{ si } k = 2 \\
    0.4 \text{ si } k = 3
\end{cases}
$$
Sea $X_0=0$, y sea $X_n= max\left\{\xi_1, \ldots,\xi_n\right\}$ el $\xi$ máximo observado hasta la fecha. Demuestra que $\left\{X_n\right\}$ es una cadena de Markov y determina la matriz de transición.
\item Sea $\left\{X_n\right\}$ una cadena de Markov con espacio de estados $S = \left\{0,1,2\right\}$ y matriz de transición:
$$P= \left[
    \begin{matrix}
        0.1 & & 0.2 & 0.7 \\
        0.2 & & 0.2 & 0.6 \\
        0.6 & & 0.1 & 0.3 \\
    \end{matrix}
    \right]
$$
\begin{enumerate}
    \item Calcula la matriz de transición $P^{(2)}$ en dos pasos.
    \item Calcula: $P\left(\left. X_3 = 1\right\vert X_1=0\right)$
    \item Calcula: $P\left(\left. X_3 = 1\right\vert X_0=0\right)$
\end{enumerate}
\item Sea $\left\{X_n\right\}$ una cadena de Markov con matriz de transición:
$$P= \left[
    \begin{matrix}
        0.6 & & 0.3 & 0.1 \\
        0.3 & & 0.3 & 0.4 \\
        0.4 & & 0.1 & 0.5 \\
    \end{matrix}
    \right]
$$
Si se sabe que $X_0=1$, determina $P(X_2 = 1)$.
\item Supongamos que $\left\{X_n\right\}$ es una cadena de Markov con dos estados y matriz de transición:
$$P= \left[
    \begin{matrix}
        \alpha & 1-\alpha \\
        1-\beta & \beta \\
    \end{matrix}
    \right]
$$
Demuestra que el proceso $\left\{Z_n\right\}= (X_{n-1}, X_n)$ con estados $(0,0)$, $(0,1)$, $(1,0)$, $(1,1)$, es una cadena de Markov. Determian su matriz de transición.
\item Sea $\left\{X_n\right\}$ una cadena de Markov. Muestra que
$$
P\left(\left.X_0=x_0\right\vert X_1 = x_1, \ldots, X_n = x_n\right) = P\left(\left. X_0=x_0 \right\vert X_1 = x_1\right)
$$
\item Sea $\left\{X_n\right\}$ una cadena de Markov. Muestra que
$$
P\left(\left.X_{n+m}=x_{n+m}, \ldots, X_{n+1} = x_{n+1}\right\vert X_n = x_n\right) = P\left(\left.X_{m}=x_{n+m}, \ldots, X_{1} = x_{n+1}\right\vert X_0 = x_n\right)
$$
\end{enumerate}
Esto lo escribo desde VSCode web
\end{document}